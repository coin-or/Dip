We will use the Bin Packing Problem to demonstrate the implementation of customised branching rules, custom cuts, heuristics, and a column generation algorithm.

The solution of the problem determines which, of $m$ bins, to use and also places $n$ items of various sizes into the bins in a way that (in this version) minimises the wasted capacity of the bins.
Each item $j=1, \ldots, n$ has a size $s_j$ and each bin has capacity $C$.
Extensions of this problem arise often in \ac{MILP} in problems including network design and rostering.

The \ac{MILP} formulation of the bin packing problem is straightforward.
The decision variables are
\begin{align*}
x_{ij} &= \begin{cases} 1 & \text{if item $j$ is placed in bin $i$} \\
0 & \text{otherwise} \end{cases} \\
y_i &= \begin{cases} 1 & \text{if bin $i$ is used} \\
0 & \text{otherwise} \end{cases} \\
w_i &= \text{ ``wasted'' capacity in bin $i$}
\end{align*}
and the formulation is
\[
\begin{array}{rr@{\ }ll}
       \min & \displaystyle \sum_{i=1}^m w_i \\
\text{s.t.} & \displaystyle \sum_{i=1}^m x_{ij}           & = 1, j = 1, \ldots, n      & \text{ (each item packed)} \\
            & \displaystyle \sum_{j=1}^n s_j x_{ij} + w_i & = C y_i, i = 1, \ldots, m  & \text{ (aggregate packing for bin $i$)} \\
            & \multicolumn{2}{l}{x_{ij} \leq y_i, i = 1, \ldots, m, j = 1, \ldots, n}  & \text{ (individual packing for bin $i$)} \\[6pt]
            & \multicolumn{3}{l}{x_{ij} \in \{ 0, 1\}, w_i \geq 0, y_i \in \{0, 1\}, i = 1, \ldots, m, j = 1, \ldots, n}
\end{array}
\]

Note that the constraints for the individual packing in a bin are not necessary for defining the solution, but tighten the \ac{MILP} formulation by removing fractional solutions from the solution space. Before looking at the advanced techniques that can be easily implemented using Dippy, we will examine how to formulate the bin packing problem in PuLP and Dippy.

\subsection{Formulating the Bin Packing Problem} \label{sbs:formulate}

Before formulating we need to include the PuLP and Dippy modules into Python
\lstinputlisting[firstnumber=3,linerange=3-21]{../../examples/Dippy/bpp/bin_pack_func.py}
and define a class to hold a bin packing problem's data
\lstinputlisting[firstnumber=25,linerange=25-31]{../../examples/Dippy/bpp/bin_pack_func.py}

The \lstinline{formulate} function is defined with a bin packing problem object as input and creates a \lstinline{DipProblem} (with some display options defined)
\lstinputlisting[firstnumber=33,linerange=33-38]{../../examples/Dippy/bpp/bin_pack_func.py}

Then, using the bin packing problem object's data (i.e., the data defined within \lstinline{bpp}), the decision variables
\lstinputlisting[firstnumber=40,linerange=40-45]{../../examples/Dippy/bpp/bin_pack_func.py}
objective function
\lstinputlisting[firstnumber=47,linerange=47-47]{../../examples/Dippy/bpp/bin_pack_func.py}
\newpage
and constraints are defined
\lstinputlisting[firstnumber=49,linerange=49-59]{../../examples/Dippy/bpp/bin_pack_func.py}

Finally, the bin packing problem object and the decision variables are all ``embedded'' within the \lstinline{DipProblem} object, \lstinline{prob}, and this object is returned (note that the objective function and constraints could also be similarly embedded)
\lstinputlisting[firstnumber=64,linerange=64-71]{../../examples/Dippy/bpp/bin_pack_func.py}

In order to solve the bin packing problem, only the \lstinline{DipProblem} object, \lstinline{prob}, is required (note that no \lstinline{dippyOpts} are specified, so the Dippy defaults are used)
\lstinputlisting[firstnumber=73,linerange={100-101,108-108,115-122}]{../../examples/Dippy/bpp/bin_pack_func.py}

To solve an instance of the bin packing problem, the data needs to be specified and then the problem formulated and solved
\lstinputlisting[firstnumber=3,linerange=3-11]{../../examples/Dippy/bpp/bin_pack_instance.py}
\lstinputlisting[firstnumber=13,linerange=13-21]{../../examples/Dippy/bpp/bin_pack_instance.py}

Solving this bin packing problem instance in Dippy gives the branch-and-bound tree shown in figure \ref{fig:bpp_tree1} (note thet the integer solution found -- indicated in blue \lstinline{S: 5.0} -- bounds all other nodes in the tree) with the final solution packing items 1 and 2 into bin 0 (for a waste of 1), items 3 and 5 into bin 1 (for a waste of 3) and item 4 into bin 3 (for a waste of 1).
\begin{figure}[htp]
\begin{center}
%\includegraphics[bb=0 0 815 496,scale=0.50]{img/bpp_tree1.png}
\includegraphics[scale=0.16]{img/bpp_tree1.eps}
\end{center}
\caption{Branch-and-bound tree for bin packing problem instance.} \label{fig:bpp_tree1}
\end{figure}

\subsection{Adding Customised Branching} \label{sbs:branch}
In \scnref{sbs:callbacks} we explained the modifications to \ac{DIP} (\texttt{chooseBranchVar} to \\ \texttt{chooseBranchSet}) and also how to implement a simple variable branch using \texttt{chooseBranchSet}. However, other branching methods may use advanced branching techniques such as the ones demonstrated in the remainder of this section.

From \ac{DIP}, \texttt{chooseBranchSet} calls \texttt{branch\_method} in Dippy. In Dippy, we can implement customised branching by defining a \texttt{branch\_method}. The function \texttt{branch\_method} has two inputs supplied by \ac{DIP}:
\begin{enumerate}
\item \texttt{prob} -- the \texttt{DipProblem} being solved;
\item \texttt{sol} -- an indexable object representing the solution at the current node.
\end{enumerate}
We define \texttt{branch\_method} using these inputs in the same Python file as the model definition. Note that Dippy can access the variables from the original formulation. Due to Python's scoping rules, no complicated indexing or searching is required (also note Python starts its array indexing at 0 -- cf. C/C++).

\subsection{Advanced Branching in the Capacitated Facility Location problem} \label{sbs:fac_brch}

When solving the facility location problem (see \sbsref{sbs:facility}) one difficulty is symmetry in the solution space. Since the facilities are identical, solvers consider multiple solutions that differ only in the labelling of the facility locations. To overcome this constraints for ordering the facility locations can be included:
\[ y_i \geq y_{i+1}, i = 1, \ldots, m-1 \]
\lstinputlisting[firstnumber=44,linerange=44-47]{C:/COIN/Dippy/examples/facility.py}

These ordering constraints also introduce the opportunity to implement an effective branch on the number of facilities. If $\displaystyle\sum_{i=1}^m y_i = \alpha \notin \mathbb{Z}$, then:
\vspace*{-6pt}
\begin{center}
\begin{tabular}{l|l}
the branch down restricts & the branch up restricts \\
$\displaystyle\sum_{i=1}^m y_i \leq \lfloor \alpha \rfloor$ &
$\displaystyle\sum_{i=1}^m y_i \geq \lceil \alpha \rceil$ \\
and the ordering means that & and the ordering means that \\
$y_i = 0, i = \lceil \alpha \rceil, \ldots, m$ &
$y_i = 1, i = 1, \ldots, \lceil \alpha \rceil$
\end{tabular}
\end{center}

To implement this branch in Dippy simply requires the definition of the \\ \texttt{branch\_method}. Remember Python's scoping rules allow the model structures to be accessed directly within our customised branching function, no complicated indexing or searching is necessary.
\lstinputlisting[firstnumber=49,linerange=49-65]{C:/COIN/Dippy/examples/facility.py}

By only adding the ordering constraints we improve the solution time from 1.17s to 0.28s of CPU time and decrease the tree size from 375 to 76 nodes. By adding the advanced branching we decrease the solution time further to 0.05s of CPU time and the tree size decreases to 3 nodes. A full summary of the effect of advanced \ac{MILP} methods, including branching is given in \scnref{scn:concl}.

\subsection{Advanced Branching in the Coke Supply Chain problem}

In the Coke Supply Chain problem, the main source of fractionality is the size of the plant facilities. If the necessary size to process coal to coke is between the given sizes then the \texttt{buildVars} variable will be fractional. We can define a branch that forces the plant to be either less than the nearest given size below or greater than the nearest given size above. For example, if the size = 210, then it lies between the given sizes of 150 and 225, so the branch down keeps the size $\leq$ 150 and the branch up keeps the size $\geq$ 225. However, branching on the size of the plant itself does not help reduce fractionality as the fractional amounts simply change to match the new plant size. The most effective way to enforce the branch and reduce fractionality is by banning infeasible variables. In our example, in the down branch all buildVars for the location with capacity $>$ 150 are set to 0 (by setting the upper bound to 0) and in the up branch all buildVars for the location with capacity $<$ 255 are set to 0 (also by setting the upper bound to 0). We can define this branching in Dippy.
\lstinputlisting[firstnumber=135,linerange=135-158]{C:/COIN/Dippy/examples/coke.py}

With this advanced branching the solution time decreases from 1.09s to 0.62s of CPU time and the tree reduces from 201 nodes to 76 nodes.

\begin{comment}

\subsection{Advanced Branching in the Wedding Planner problem} \label{sbs:wed_brch}

When solving the Wedding Planner problem using \ac{MILP}, guests will often be ``split'' between tables. Branching a particular $x_{gt}$ to 0 is not effective as linear programming can simply swap entire tables and maintain the split on a different table. One way to strengthen branching if some $x_{gt}$ is fractional is to: 1) find the minimum index table $t$ with $x_{gt} > 0$; 2) define a ``down'' branch that restricts guest $g$ to sit at tables with index $t$ or smaller; 3) define a ``up'' branch that restricts guest $g$ to sit at tables with index $t + 1$ or bigger. This method creates more balanced branches and reduces the size of the branch-and-bound tree.
\lstinputlisting[firstnumber=77,linerange=77-90]{C:/COIN/Dippy/examples/wedding_dip.py}

With this advanced branching the solution times get worse and the tree size increases! Not a good branch...

\end{comment}




\subsection{Adding Customised Cut Generation} \label{sbs:cuts}
\acrodef{CGL}{Cut Generation Library}
\begin{sloppypar} By default \ac{DIP} uses the \ac{CGL} to add cuts. We can use \lstinline{dippyOpts} to turn off \ac{CGL} cuts and observe how effective the \ac{CGL} are
\lstinputlisting[firstnumber=73,linerange={107-108,110-110,115-117}]{../../examples/Dippy/bpp/bin_pack_func.py}
The branch-and-bound tree is significantly larger (see figure \ref{fig:bpp_tree4}) than the original branch-and-bound tree that only used \ac{CGL} cuts (see figure \ref{fig:bpp_tree1}).
\begin{figure}[htp]
\begin{center}
\includegraphics[scale=0.075]{img/bpp_tree4.eps}
\end{center}
\caption{Branch-and-bound tree for bin packing problem instance without \ac{CGL} cuts.} \label{fig:bpp_tree4}
\end{figure}


To add user-defined cuts in Dippy, we first define a new procedure for generating cuts and (if necessary) a procedure for determining a feasible solution.
Within Dippy, this requires two new functions, \lstinline{generate_cuts} and \lstinline{is_solution_feasible}.
As in \sbsref{sbs:branch}, the embedded bin packing problem and decisions variables make it easy to access the solution values of variables in the bin packing problem.
The inputs to \lstinline{is_solution_feasible} are:
\begin{enumerate}
\item \lstinline{prob} -- the \lstinline{DipProblem} being solved;
\item \lstinline{sol} -- an indexable object representing the solution at the current node;
\item \lstinline{tol} -- the zero tolerance value.
\end{enumerate}
and the inputs to \lstinline{generate_cuts} are:
\begin{enumerate}
\item \lstinline{prob} -- the \lstinline{DipProblem} being solved;
\item \lstinline{node} -- various properties of the current node, including the solution.
\end{enumerate}

If a solution is determined to be infeasible either by \ac{DIP} (for example some integer variables are fractional) or by \lstinline{is_solution_feasible} (which is useful for solving problems like the travelling salesman problem with cutting plane methods), cuts will be generated by \lstinline{generate_cuts} and the in-built \ac{CGL} (if enabled).\end{sloppypar}

\begin{comment}

Marchand and Wolsey \cite{agg_mir2001} define many types of cuts for \ac{MILP} problems.
One of these is the \textit{weighted inequality}.
For each facility location $i$ and some subset $S_i (\subseteq \{1, \ldots, n \})$ of the products we can calculate
\[ \mu_i = C - \sum_{j \in S_i} w_j x_{ij} \]
and use it to generate a weighted inequality
\[ \sum_{j \in S_i} w_j x_{ij} + \sum_{j \notin S_i} (w_j - \mu_i)^+ x_{ij} \leq C - \mu _i \]
which forms a valid inequality for the facility location problem.

The cut generating function creates the subsets $S_i$ for each location from the fractional solution in a greedy way depending on the $x_{ij}$ values, and from these we generate a set of weighted inequality cuts.
The code listing below shows how to build the set of cuts, and omits the generation of $S_i$ for the sake of brevity.
\lstinputlisting[firstnumber=67,linerange=67-74]{../../examples/Dippy/bpp/bin_pack_func.py}
$\vdots$\newpage
\lstinputlisting[firstnumber=98,linerange=98-111]{../../examples/Dippy/bpp/bin_pack_func.py}

Adding the weighted inequality cut generator reduces the branch-and-bound tree size from 419 nodes to 77 nodes.

\end{comment}

\subsection{Adding Customised Column Generation} \label{sbs:column}
Using Dippy it is easy to transform a problem into a form that can be solved by either branch-and-cut or branch-price-and-cut.
Branch-price-and-cut decomposes a problem into a master problem and a number of distinct subproblems.
We can identify subproblems using the \lstinline{relaxation} member of the \lstinline{DipProblem} class.
Once the subproblems have been identified, then they can either be ignored (when using branch-and-cut -- the default method for \ac{DIP}) or utilised (when using branch-price-and-cut -- specified by turning on the \lstinline{doPriceCut} option).

In branch-price-and-cut, the original problem is decomposed into a master problem and multiple subproblems~\cite{DWDecomp00}:
\begin{equation}
\begin{array}{rr@{\ }r@{\ }r@{\ }r@{\ }l}
             \min & c_1^\top x_1 & + \ c_2^\top x_2 & + \ \cdots & + \ c_K^\top x_K \\
\text{subject to} & A_1 x_1      & + \ A_2 x_2      & + \ \cdots & + \ A_K x_K      & = b \\
                  &              &   F_2 x_2      &          &                & = f_2 \\
                  &              &                &  \ddots  &                & \ \ \vdots \\
                  &              &                &          &   F_K x_K      & = f_K \\
                  & x_1 \in \mathbb{Z}^{+}_{n_1} &, x_2 \in \mathbb{Z}^{+}_{n_2}&, \ldots, x_K & \in \mathbb{Z}^{+}_{n_K} \quad
\end{array}
\label{eqn:decomp}
\end{equation}

In \eqref{eqn:decomp}, there are $K-1$ subproblems defined by the constraints $F_k x_k = f_k, k \in 2, \ldots, K$. The constraints $A_1 x_1 + A_2 x_2 + \cdots + A_K x_K = b$ are known as \textit{linking} constraints. Instead of solving \eqref{eqn:decomp} directly, column generation uses a convex combination of solutions $y^k$ to each subproblem $j$ to define the subproblem variables:
\begin{equation}
x_k = \sum_{l_k=1}^{L_k} \lambda^k_{l_k} y^k_{l_k} \label{eqn:combin}
\end{equation}
where $0 \leq \lambda^k_{l_k} \leq 1$ and $\sum_{l_k=1}^{L_k} \lambda^k_{l_k} = 1$. By substituting \eqref{eqn:combin} into the linking constraints and recognising that each $y^k_{l_k}$ satisfies $F_k x_k = f_k, x_k \in \mathbb{Z}^{+}_{n_k}$ (as it is a solution of this subproblem), we can form the \textit{restricted} master problem (RMP) with corresponding duals ($\pi$, $\gamma_1, \ldots, \gamma_K$):
\begin{equation}
\begin{array}{rr@{\ }r@{\ }r@{\ }r@{\ }ll}
             \min & c_1^\top x_1 & + \displaystyle\sum_{l_2=1}^{L_2} \left(c_2^\top y^2_{l_2} \right) \lambda^2_{l_2} & + \ \cdots & + \displaystyle\sum_{l_K=1}^{L_K} \left(c_K^\top y^K_{l_K} \right) \lambda^K_{l_K} \\
\text{subject to} & A_1 x_1      & + \displaystyle\sum_{l_2=1}^{L_2} \left(A_2 y^2_{l_2} \right) \lambda^2_{l_2} & + \ \cdots & + \displaystyle\sum_{l_K=1}^{L_K} \left( A_K y^K_{l_K} \right) \lambda^K_{l_K}      & = b & : \pi \\
                  &              &   \displaystyle\sum_{l_2=1}^{L_2} \lambda^2_{l_2}      &          &                & = 1 & : \gamma_1 \\
                  &              &                &  \ddots  &                & \ \ \vdots \\
                  &              &                &          &   \displaystyle\sum_{l_K=1}^{L_K} \lambda^K_{l_K}      & = 1 & : \gamma_K \\
                  &              &   \displaystyle\sum_{l_2=1}^{L_2} y^2_{l_2} \lambda^2_{l_2}      &          &                & \in \mathbb{Z}^{+}_{n_2} \\
                  &              &                &  \ddots  &                & \ \ \vdots \\
                  &              &                &          &   \displaystyle\sum_{l_K=1}^{L_K} y^K_{l_K} \lambda^K_{l_K}      & \in \mathbb{Z}^{+}_{n_K} \\
                  &          x_1 & \in \mathbb{Z}^{+}_{n_1}, \lambda^2 \in [0, 1]_{L_2}, & \ldots, \lambda^K & \in [0, 1]_{L_K} \hspace{1.25cm} &
\end{array}
\label{eqn:rmp}
\end{equation}
The RMP provides the optimal solution $x^*_1, x^*_2, \ldots, x^*_K$ to the original problem \eqref{eqn:decomp} if the necessary subproblem solutions are present in the RMP. That is, if $y^{k,*}_{l_k}, l_k =1, \ldots, L_k, k = 2, \ldots K$ such that $x^*_k = \sum_{l_k=1}^{L_k} \lambda^k_{l_k} y^{k,*}_{l_k}, k = 2, \ldots, K$ have been included.

Given that $x^*_k, k = 1, \ldots, K$ are not known a priori, column generation starts with an initial solution consisting of $x_1$ and initial sets of subproblem solutions. ``Useful'' subproblem solutions, that form columns for the RMP, are found by looking for subproblem solutions that provide columns with negative reduced cost. The reduced cost of a solution $y^k_{l_k}$'s column, i.e., the reduced cost for $\lambda^k_{l_k}$,  is given by $c_k^\top y^k_{l_k} - \pi^\top A_k y^k_{l_k} - \gamma_k$. To find a solution with minimum reduced cost we can solve:
\begin{equation}
\begin{array}{rr@{\,}ll}
{\cal S}_k: \min & (c_k - \pi^\top A_k)^\top &x_k - \gamma_k & \text{(reduced cost for corresponding $\lambda^k$)} \\
\text{subject to} & F_k & x_k      = f_k & \text{(ensures that $y^k$ solves subproblem $k$)} \\
                  & & x_k \in \mathbb{Z}^{+}_{n_k}
\end{array}
\label{eqn:subprob}
\end{equation}
If the objective value of ${\cal S}_k$ is less than $0$, then the solution $y^k$ will form a column in the RMP whose inclusion in the basis would improve the objective value of the RMP. The solution $y^k$ is added to the set of solution used in the RMP. There are other mechanisms for managing the sets of solutions present in \ac{DIP}, but they are beyond the scope of this paper.

Within \ac{DIP}, hence Dippy, the RMP and \textit{relaxed} problems $S_k, k = 2, \ldots, K$ are not specified explicitly. Rather, the constraints for each subproblem $F_k x_k = f_k$ are specified by using the \lstinline{.relaxation[j]} syntax. \ac{DIP} then automatically constructs the RMP and the relaxed problems $S_k, k = 2, \ldots, K$. The relaxed subproblems $S_k, k = 2, \ldots, K$ can either be solved using the default \ac{MILP} solver (CBC) or a customised solver. A customised solver can be defined by the \lstinline{relaxed_solver} function.
This function has 4 inputs:
\begin{enumerate}
\item \lstinline{prob} -- the \lstinline{DipProblem} being solved;
\item \lstinline{index} -- the index $k$ of the subproblem being solved;
\item \lstinline{redCosts} -- the reduced costs for the $x_k$ variables $c_k - \pi^\top A_k$;
\item \lstinline{convexDual} -- the dual value for the convexity constraint for this subproblem $\gamma_k$.
\end{enumerate}
In addition to subproblem solutions generated using RMP dual values, initial columns for subproblems can also be generated either automatically using CBC or using a customised approach.
A customised approach to initial variable generation can be defined by the \lstinline{init_vars} function.
This function has only 1 input, \lstinline{prob}, the \lstinline{DipProblem} being solved.

Starting from the original capacitated facility location problem from \scnref{scn:techs}:
\[
\begin{array}{rr@{\ }ll}
       \min & \displaystyle \sum_{i=1}^m w_i \\
\text{s.t.} & \displaystyle \sum_{i=1}^m x_{ij}           & = 1, j = 1, \ldots, n      & \text{ (each product produced)} \\
            & \displaystyle \sum_{j=1}^n r_j x_{ij} + w_i & = C y_i, i = 1, \ldots, m  & \text{ (aggregate capacity at location $i$)} \\
            & \multicolumn{2}{l}{x_{ij} \leq y_i, i = 1, \ldots, m, j = 1, \ldots, n}  & \text{ (disaggregate capacity at location $i$)} \\[6pt]
            & \multicolumn{3}{l}{x_{ij} \in \{ 0, 1\}, w_i \geq 0, y_i \in \{0, 1\}, i = 1, \ldots, m, j = 1, \ldots, n}
\end{array}
\]
we can decompose this formulation:
\[
\begin{array}{r@{\ }r@{\ }r@{\ }r@{\ }c@{\ }r@{\ }l@{\ }l}
%      r           r                 r         r          c             r               l        
\min        &                     &        &  1 w_2 & \cdots &                     &  +1 w_m 				\\
\text{s.t.} & I \mathbf{x}_2      &        &        & \cdots & + I \mathbf{x}_m    &         &= 1 \text{ (each product produced)} \\
            & r^\top \mathbf{x}_2 & -C y_2 & +1 w_2 &        &                     &         &= 0 \text{ (aggregate cap. at loc. 2)} \\
            & I \mathbf{x}_2      & -e y_2 &        &        &                     &         &\leq 0 \text{ (disaggregate cap. at loc. 2)}\\
            &                     &        &        & \ddots \\
            &                     &        &        &        & r^\top \mathbf{x}_m & -C y_m +1 w_m  & = 0 \text{ (aggregate cap. at loc. K)}\\
            &                     &        &        &        & + I \mathbf{x}_m    & -e y_m         & \leq 0 \text{ (disaggregate cap. at loc. K)}
\end{array}
\]
where
\[
\mathbf{x}_i = \begin{pmatrix}
x_{i1} \\ \vdots \\ x_{in}
\end{pmatrix}, r = \begin{pmatrix}
r_{1} \\ \vdots \\ r_{n}
\end{pmatrix} \text{ and }
e = \begin{pmatrix}
1 \\ \vdots \\ 1
\end{pmatrix}.
\]
Now the subproblems $F_k x_k = f_k, k = 2, \ldots, K$ are
\[
\begin{bmatrix}
r^\top & -C & 1 \\
I & e 
\end{bmatrix} \begin{bmatrix}
\mathbf{x}_i \\
y_i \\
w_i
\end{bmatrix}
\begin{matrix}
= \\ \leq
\end{matrix}
\begin{bmatrix}
0 \\ 0
\end{bmatrix},
\]
\[ c_k^\top = \left[ \begin{array}{c|c|c} 0 & 0 & 1 \end{array} \right], A_k = \left[ \begin{array}{c|c|c} I & 0 & 0 \end{array} \right], \]
so ${\cal S}_k$ becomes
\[
\begin{array}{rr@{\ }r@{\ }r@{\ }l}
{\cal S}_i: \min  & \sum_{j=1}^n -\pi_j x_{ij} \, &           & +1 w_i & - \gamma_i \\
\text{subject to} & \sum_{j=1}^n r_j    x_{ij} \, & -C y_i \, & +1 w_i & = 0 \\
                  &                     x_{ij} \, & -  y_i \, &        & \leq 0, j = 1, \ldots, n \\
                  &                     x_{ij},   &    y_i,   & \in \{0, 1 \}, & j = 1, \ldots, n, w_i \geq 0
\end{array}
\]
where $\pi_j$ is the dual variable for the assignment constraint for product $j$ in the RMP.

\begin{sloppypar}In Dippy, we define subproblems for each facility location using the \lstinline{.relaxation} syntax for the aggregate and disaggregate capacity constraints:\end{sloppypar}
\lstinputlisting[firstnumber=32,linerange=32-41]{../../examples/Dippy/bpp/bin_pack_decomp_func.py}

All remaining constraints (the assignment constraints that ensure each product is assigned to a facility) form the master problem when using branch-price-and-cut.
To use branch-price-and-cut we turn on the \lstinline{doPriceCut} option:
\lstinputlisting[firstnumber=206,linerange=206-209]{../../examples/Dippy/bpp/bin_pack_decomp_func.py}

Note that symmetry is also present in the decomposed problem, so we add ordering constraints (described in \sbsref{sbs:branch}) to the RMP :
\lstinputlisting[firstnumber=43,linerange=43-46]{../../examples/Dippy/bpp/bin_pack_decomp_func.py}

\begin{sloppypar}Using branch-price-and-cut, the RMP takes about ten times as long to solve as the original formulation, and has a search tree size of 37 nodes.
The \lstinline{generateInitVars} option uses CBC by default to find initial columns for the RMP and then uses CBC to solve the relaxed problems.
Dippy lets us provide our own approaches to solving the relaxed problems and generating initial variables, which may be able to speed up the overall solution process.\end{sloppypar}

In the relaxed problem for location $i$, the objective simplified to $\min \sum_{j=1}^n -\pi_j x_{ij} +1 w_i - \gamma_i$. 
However, the addition of the ordering constraints and the possibility of a Phase I/Phase II approach in the \ac{MILP} solution process to find initial variables mean that our method must work for any reduced costs, i.e., the objective becomes $\min \sum_{j=1}^n d_j x_{ij} + f y_i + g w_i - \gamma_i$. Although the objective changes, the constraints remain the same. If we choose not to use a location, then $x_{ij} = y_i = w_i = 0$ for $j=1, \ldots, n$ and the objective is $-\gamma_i$. Otherwise, we use the location and $y_i = 1$ and add $f$ to the objective. The relaxed problem reduces to:
\[
\begin{array}{rr@{\ }r@{\ }l}
\min              & \sum_{j=1}^n d_j x_{ij} \, & +g w_i & - \gamma_i \\
\text{subject to} & \sum_{j=1}^n r_j x_{ij} \, & +1 w_i & = C \\
                  &                  x_{ij},   &    w_i    & \in \{0, 1 \}, j = 1, \ldots, n
\end{array}
\]
However, the constraint ensures $w_i = C - \sum_{j=1}^n r_j x_{ij}$, so we can reformulate as:
\[
\begin{array}{rr@{\ }l}
\min              & \sum_{j=1}^n (d_j - g r_j) x_{ij} \, & +f C - \gamma_i \\
\text{subject to} & C - \sum_{j=1}^n r_j x_{ij} & \geq 0 \Rightarrow \sum_{j=1}^n r_j x_{ij} \leq C \\
                  &                  x_{ij},   & \in \{0, 1 \}, j = 1, \ldots, n
\end{array}
\]
This is a 0-1 knapsack problem with ``effective costs'' costs for each product $j$ of $d_j - g r_j$. We can use dynamic programming to find the optimal solution.

In Dippy, we can access the problem data, variables and their reduced costs, so the 0-1 knapsack dynamic programming solution is straightforward to implement and use:
\lstinputlisting[firstnumber=66,linerange=66-81]{../../examples/Dippy/bpp/bin_pack_decomp_func.py}
$\vdots$\newpage
\lstinputlisting[firstnumber=83,linerange=83-105]{../../examples/Dippy/bpp/bin_pack_decomp_func.py}

Adding this customised solver reduces the solution time because it has the benefit of knowing it is solving a knapsack problem rather than a general \ac{MILP}.

To generate initial facilities (complete with assigned products) we implemented two approaches.
The first approach used a first-fit method and considered the products in order of decreasing requirement:
\lstinputlisting[firstnumber=146,linerange=146-169]{../../examples/Dippy/bpp/bin_pack_decomp_func.py}
\newpage
\lstinputlisting[firstnumber=172,linerange=172-184]{../../examples/Dippy/bpp/bin_pack_decomp_func.py}
The second approach simply assigned one product to each facility:
\lstinputlisting[firstnumber=186,linerange=186-197]{../../examples/Dippy/bpp/bin_pack_decomp_func.py}

Using Dippy we can define both approaches at once and then define which one to use by setting the \lstinline{init_vars} method:
\lstinputlisting[firstnumber=199,linerange=199-200]{../../examples/Dippy/bpp/bin_pack_decomp_func.py}

These approaches define the initial sets of subproblem solutions $y^k_{l_k}, l_k=1,$ $\ldots, L_k, k = 1, \ldots, K$ for the initial RMP before the relaxed problems are solved using the RMP duals.

The effect of the different combinations of column generation, customised subproblem solvers and initial variable generation methods, both by themselves and combined with branching, heuristics, etc are summarised in \tabref{tab:fac_exp}. For this size of problem, column generation does not reduce the solution time significantly (if at all). However, we show in \scnref{scn:concl} that using column branching enables \ac{DIP} (via Dippy and PuLP) to be competitive with state-of-the-art solvers.

\newpage\subsection{Adding Customised Heuristics} \label{sbs:heuristics}
To add user-defined heuristics in Dippy, we first define a new procedure for node heuristics, \texttt{heuristics}. This function has three inputs:
\begin{enumerate}
\item \texttt{prob} -- the \texttt{DipProblem} being solved;
\item \texttt{xhat} -- an indexable object representing the fraction solution at the current node;
\item \texttt{cost} -- the objective coefficients of the variables.
\end{enumerate}
Multiple heuristics can be executed and all heuristic solutions can be returned to \ac{DIP}. Different problems benefit from different heuristic approaches and a heuristic that solves the original problem may not be as useful when a fractional solution is available. We show how solve a heuristic for the overall problem and also how to implement a heuristic for fractional solutions. As in \scnref{scn:branch}, Python's scoping rules allow us to easily access the solution values of variables in our problem.

\subsection{Customised Heuristics for the Capacitated Facility Location problem} \label{sbs:fac_heur}

An initial allocation of production to locations can be found heuristically using the same first-fit heuristic that provided initial solutions for the column generation approach (see \sbsref{sbs:fac_cols}). The first-fit heuristic iterates through the items requiring production and the facility locations allocating production at the first facility that has sufficient capacity to produce the item.
\lstinputlisting[firstnumber=116,linerange=116-127]{C:/COIN/Dippy/examples/facility.py}

\vfill
\newpage
\lstinputlisting[firstnumber=129,linerange=129-144]{C:/COIN/Dippy/examples/facility.py}

The first-fit heuristic can then be used to provide an initial, feasible solution at the root node within the customised \texttt{heuristics} function (see lines 215-220):
\lstinputlisting[firstnumber=211,linerange=211-230]{C:/COIN/Dippy/examples/facility.py}

Running the first-fit heuristic before starting the branching process increases the solution time from 1.17s to 1.48s of CPU time and the number of nodes in the search tree from 375 nodes to 399 nodes.

\newpage

At each node in the branch-and-bound tree, the fractional solution (provided by \texttt{xhat}) gives an indication of the best allocation of production, albeit fractional. One heuristic approach to ``fixing'' the fractional solution is to consider each allocation (of an item's production to a facility) in order of decreasing fractionality and use a first-fit approach:
\lstinputlisting[firstnumber=146,linerange=146-192]{C:/COIN/Dippy/examples/facility.py}
\newpage
\lstinputlisting[firstnumber=194,linerange=194-209]{C:/COIN/Dippy/examples/facility.py}

The first-fit approach that is guided by fractional values can then be used within the \texttt{heuristics} function (see lines 225-229 in the previous listing) to create integer solutions from fractional solutions at each node.

Adding the first-fit heuristic guided by fractional values increases the solution time further from 1.48s to 1.89s of CPU time and the number of nodes remains at 399.

The reason this heuristic (in fact any heuristic) was not that helpful for this problem is that:
\begin{itemize}
\item the optimal solution is found within the first 10 nodes without any heuristics, so the heuristic only provides an improved upper bound for $< 10$ nodes;
\item the extra overhead of the heuristic at each node increases the solution time and the heuristic affects the search procedure in a way that more nodes are explored.
\end{itemize}

Heuristics generally only help in problems where feasibility is more difficult by providing upper bounds when ``fixing'' fractional solutions. In this problem, the optimal solution is found quickly and the rest of the search tree checks solutions that are symmetric.

In \scnref{scn:concl} we show the effect of the heuristic when symmetry is removed.

\begin{comment}
\subsection{Customised Heuristics for the Wedding Planner problem} \label{sbs:wed_heur}
\end{comment}

\subsection{Combining Techniques} \label{sbs:combine}

\begin{table}[ht]
%\begin{minipage}[c]{\textwidth}
%\begin{small}
\begin{center}
\begin{tabular}[c]{|lll|}
\hline
\textbf{Strategies}																 & \textbf{Time (s)}	& \textbf{Nodes} \\ 
\hline & &\\[-10pt]
Default (branch and cut)                           & 0.26							  & 419 \\
\hline
+ ordering constraints (OC)                        & 0.05 							& 77 \\
+ OC \& advanced branching (AB)                    & 0.01 							& 3 \\
\hline
+ weighted inequalities (WI)                       & 0.34 							& 77 \\
+ WI \& OC                                         & 0.17 							& 20 \\
+ WI \& OC \& AB                                   & 0.06 							& 4 \\
\hline
+ first-fit heuristic (FF) at root node            & 0.28 							& 419 \\
+ FF \& OC                                         & 0.05 							& 77 \\
+ FF \& OC \& AB                                   & 0.01 							& 3 \\
\hline
+ FF \& WI                                         & 0.36 							& 77 \\
+ FF \& WI \& OC                                   & 0.14 							& 17 \\
+ FF \& WI \& OC \& AB                             & 0.05 							& 3 \\
\hline
+ fractional-fit heuristic (RF) at nodes           & 0.28 							& 419 \\
+ RF \& OC                                         & 0.05 							& 77 \\
+ RF \& OC \& AB                                   & 0.01 							& 3 \\
\hline
+ WI \& RF                                         & 0.38 							& 77 \\
+ WI \& RF \& OC                                   & 0.14 							& 17 \\
+ WI \& RF \& OC \& AB                             & 0.05 							& 3 \\
\hline
+ FF \& RF                                         & 0.28 							& 419 \\
+ FF \& RF \& OC                                   & 0.05 							& 77 \\
+ FF \& RF \& OC \& AB                             & 0.01 							& 3 \\
\hline
+ WI \& FF \& RF                                   & 0.38 							& 77 \\
+ WI \& FF \& RF \& OC                             & 0.14 							& 17 \\
+ WI \& FF \& RF \& OC \& AB                       & 0.05 							& 3 \\
\hline
+ column generation (CG)                           & 2.98 							& 37 \\
+ CG \& OC                                         & 2.07 							& 23 \\
+ CG \& OC \& AB                                   & 0.56								& 10 \\
\hline
+ CG \& customised subproblem solver (CS)          & 2.87 							& 37 \\
+ CG \& CS \& OC                                   & 1.95 							& 23 \\
+ CG \& CS \& OC \& AB                             & 0.44 							& 10 \\
\hline
+ CG \& first-fit initial variable generation (FV) & 3.96 							& 45 \\
+ CG \& CS \& FV                                   & 3.72 							& 45 \\
+ CG \& CS \& FV \& OC                             & 1.70 							& 18 \\
+ CG \& CS \& FV \& OC \& AB                       & 0.22 							& 3\\
\hline
+ CG \& one-each initial variable generation (OV)  & 3.40 							& 41 \\
+ CG \& CS \& OV                                   & 3.33 							& 41 \\
+ CG \& CS \& OV \& OC                             & 2.23 							& 24 \\
+ CG \& CS \& OV \& OC \& AB                       & 0.27 							& 3 \\
\hline
\end{tabular} 
\end{center}
%\end{small}
%\end{minipage}
\caption{Experiments for the Capacitated Facility Location Problem} \label{tab:fac_exp}
\end{table}

The techniques and modifications of the solver framework can be combined to improve performance further.
\Tabref{tab:fac_exp} shows that it is possible to quickly and easily test many approaches for a particular problem, including combinations of approaches\footnote{All tests were run using Python 2.7.1 on a Windows~7 machine with an Intel Core~2~Duo T9500@2.60GHz CPU.}.
Looking at the results shows that the heuristics only help when the size of the branch-and-bound tree has been reduced with other approaches, such as ordering constraints and advanced branching.
Approaches for solving this problem that warrant further investigation use column generation, the customised solver and either ordering constraints or the first-fit heuristic to generate initial variables.
Tests with different data showed that the solution time for branch-price-and-cut doesn't increase with problem size as quickly as for branch-and-cut, so the column generation approaches are worth considering for larger problems.
