To add user-defined heuristics in Dippy, we first define a new procedure for node heuristics, \texttt{heuristics}. This function has three inputs:
\begin{enumerate}
\item \texttt{prob} -- the \texttt{DipProblem} being solved;
\item \texttt{xhat} -- an indexable object representing the fraction solution at the current node;
\item \texttt{cost} -- the objective coefficients of the variables.
\end{enumerate}
Multiple heuristics can be executed and all heuristic solutions can be returned to \ac{DIP}. Different problems benefit from different heuristic approaches and a heuristic that solves the original problem may not be as useful when a fractional solution is available. We show how solve a heuristic for the overall problem and also how to implement a heuristic for fractional solutions. As in \scnref{scn:branch}, Python's scoping rules allow us to easily access the solution values of variables in our problem.

\subsection{Customised Heuristics for the Capacitated Facility Location problem} \label{sbs:fac_heur}

An initial allocation of production to locations can be found heuristically using the same first-fit heuristic that provided initial solutions for the column generation approach (see \sbsref{sbs:fac_cols}). The first-fit heuristic iterates through the items requiring production and the facility locations allocating production at the first facility that has sufficient capacity to produce the item.
\lstinputlisting[firstnumber=116,linerange=116-127]{C:/COIN/Dippy/examples/facility.py}

\vfill
\newpage
\lstinputlisting[firstnumber=129,linerange=129-144]{C:/COIN/Dippy/examples/facility.py}

The first-fit heuristic can then be used to provide an initial, feasible solution at the root node within the customised \texttt{heuristics} function (see lines 215-220):
\lstinputlisting[firstnumber=211,linerange=211-230]{C:/COIN/Dippy/examples/facility.py}

Running the first-fit heuristic before starting the branching process increases the solution time from 1.17s to 1.48s of CPU time and the number of nodes in the search tree from 375 nodes to 399 nodes.

\newpage

At each node in the branch-and-bound tree, the fractional solution (provided by \texttt{xhat}) gives an indication of the best allocation of production, albeit fractional. One heuristic approach to ``fixing'' the fractional solution is to consider each allocation (of an item's production to a facility) in order of decreasing fractionality and use a first-fit approach:
\lstinputlisting[firstnumber=146,linerange=146-192]{C:/COIN/Dippy/examples/facility.py}
\newpage
\lstinputlisting[firstnumber=194,linerange=194-209]{C:/COIN/Dippy/examples/facility.py}

The first-fit approach that is guided by fractional values can then be used within the \texttt{heuristics} function (see lines 225-229 in the previous listing) to create integer solutions from fractional solutions at each node.

Adding the first-fit heuristic guided by fractional values increases the solution time further from 1.48s to 1.89s of CPU time and the number of nodes remains at 399.

The reason this heuristic (in fact any heuristic) was not that helpful for this problem is that:
\begin{itemize}
\item the optimal solution is found within the first 10 nodes without any heuristics, so the heuristic only provides an improved upper bound for $< 10$ nodes;
\item the extra overhead of the heuristic at each node increases the solution time and the heuristic affects the search procedure in a way that more nodes are explored.
\end{itemize}

Heuristics generally only help in problems where feasibility is more difficult by providing upper bounds when ``fixing'' fractional solutions. In this problem, the optimal solution is found quickly and the rest of the search tree checks solutions that are symmetric.

In \scnref{scn:concl} we show the effect of the heuristic when symmetry is removed.

\begin{comment}
\subsection{Customised Heuristics for the Wedding Planner problem} \label{sbs:wed_heur}
\end{comment}